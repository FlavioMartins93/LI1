\documentclass[a4paper]{article}

\usepackage[utf8]{inputenc}
\usepackage[portuges]{babel}
\usepackage{a4wide}

\title{Projeto de Laboratórios de Informática 1\\Grupo 104}
\author{Paulo Diogo Lourenço Martins (66655) \and Flávio Manuel Machado Martins (65277)}
\date{\today}

\pagebreak
\begin{document}

\maketitle

\pagebreak
\tableofcontents

\pagebreak
\section{Introdução}
\label{sec:intro}

Este projecto consiste no desenvolvimento duma possível implementação do clássico jogo Bomberman em battle mode. A
ideia geral do jogo é colocar bombas estrategicamente de forma a matar inimigos e destruir
obstáculos no mapa. A explosão de uma bomba pode desencadear explosões de outras
bombas, destruir obstáculos, matar inimigos ou matar o próprio jogador.

Este projecto é o primeiro trabalho prático que iremos realizar e, como tal, tem como objectivo ser uma introdução ao mundo do desenvolvimente de software,
mais especificamente, na lingunagem haskell e fazer uma introdução à utilização de técnicas e ferramentas e de desenvolvimento de software, interpretador/compilador de Haskell, 
editores de texto, sistema UNIX, sistema de controle de versões, documentação do código, técnicas de teste entre outras


\pagebreak
\section{Descrição do Problema}
\label{sec:problema}

Este projecto é divido em 6 tarefas que vão ser apresentadas a seguir.

O objectivo da \textbf{\emph{tarefa1}} é implementar um mecanismo de geração de mapas.
O input será a dimensão do mapa (número ímpar maior ou igual a 5) e um número inteiro positivo para usar como semente num gerador pseudo-aleatório. 

O objectivo da \textbf{\emph{tarefa2}} é, dada uma descrição do estado do jogo e um comando de um dos
jogadores, determinar o efeito desse comando no estado do jogo. Cada jogador é identificado
por um dígito entre 0 e 3 e os comandos podem ser os caracteres ‘U’ (ir para cima), ‘D’ (ir
para baixo), ‘L’ (ir para a esquerda), ‘R’ (ir para a direita) e ‘B’ (colocar uma bomba).

O objectivo da \textbf{\emph{tarefa3}} é, dada uma descrição do estado do jogo (idêntica à utilizada na tarefa2)
 implementar um mecanismo de compressão / descompressão que permita poupar
caracteres e, desta forma, poupar espaço em disco quando o estado do jogo for gravado
(permitindo, por exemplo, fazer pausa durante o jogo com o objectivo de o retomar mais tarde).

O objectivo da \textbf{\emph{tarefa4}} é, dada uma descrição do estado do jogo, determinar o efeito da
passagem de um instante de tempo nesse estado.

O objectivo da \textbf{\emph{tarefa5}} é implementar o jogo completo usando a biblioteca Gloss.

O objectivo da \textbf{\emph{tarefa6}} é implementar um bot que jogue Bomberman automaticamente.
O bot deve estar preparado para jogar em qualquer posição (irá receber o identificador do
jogador). O bot irá também receber o número de instantes que faltam para terminar o jogo, podendo
assim alterar a sua estratégia na parte final em que o mapa se começa a fechar.


\pagebreak
\section{Concepção da Solução}
\label{sec:solucao}

Esta secção deve descrever o trabalho efetivamente desenvolvido pelos
alunos para resolver o problema apresentado na secção
anterior. Segue-se uma sugestão de organização para esta secção.

\subsection{Tarefa1}

Para realizar esta tarefa decidimos criar o mapa em 3 fases

 - Inicialmente apenas com as suas paredes e o caracter 'v' nos locais onde iria ser preenchido com espaços livres (' ') ou tijolos ('?')

 - A seguir substituindo os 'v's pelos espaços, tijolos, bombas('+') ou flames('!') (a ideia de colocar inicialmente as bombas e flames no mapa era para facilitar a encontrar a localização das mesmas)

 - Por fim a grelha final substituindo as bombas('+') e flames(!) por tijolos


\subsection{Tarefa2}

Para realizar esta tarefa decidimos dividir o input em 5 partes de forma a facilitar o seu processamento,
sendo estas 5 partes as seguintes:

- grelha de jogo

- powerUPs bomba

- powerUPs flames

- bombas colocadas

- jogadores



Com isto facilmente desenvolvemos uma função move que,
para os comandos 'D','U','L','R' verifica se a posição para a qual o jogador se irá mover está livre,
e caso esteja ocupada retorna o input recebido, caso esteja livre move o jogador nessa direcção e
verifica se na posição destino existe algum power up e, caso exista adiciona-o ao jogador,
para o comando 'B' verifica se a posição onde está o jogador não tem nenhuma bomba colocada, 
compara-se o numero de bombas já colocadas pelo jogador com o numero de powerUps bomba que o jogador tem e, 
caso o jogador possa colocar a bomba, coloca a bomba, caso contrário devolve o input recebido


\subsection{Tarefa3}

Para comprimir o estado actual do jogo decidimos:

- Guardar a dimensao do mapa

- Remover todas as paredes da nossa grelha

- Manter as powerUps, bombas colocadas e os jogadores(sem os 2 primeiros caracteres de cada elemento e separados por vírgulas os elementos do mesmo tipo)

Criamos entao a função encode que recebe o estado de jogo actual e o devolve da seguinte forma:

-Para separar a dimensão da grelha sem paredes usamos o caracter 'd'

-Para separar a grelha sem paredes dos PU bomba usamos o caracter '+'

-Para separar os powerUP bomba dos powerUP flames usamos o caracter '!'

-Para separar os powerUP flames das bombas colocadas o caracter '*'

-Para separar as bombascolocadas dos jogadores o caracter 'j'

E de seguida uma funçao decode que recebe o estado salvo(estado de jogo codificado) e o descodifica, devolvendo ao formato original(aceite na nossa implementção do jogo).


\subsection{Tarefa4}
 
Para realizar esta Tarefa decidimos dividir a tarefa em 2 partes, simulação da explosão das bombas e similação do efeito espiral.

Para simular o efeito da explosão de bombas criamos uma função que 
simula o efeito da explosão de bombas, que explode nas quatro direcções ('D','U','R','L')
com um dado raio e que irá ser travada por qualquer parede, powerUP
 ou tijolo(removendo o tijolo e revelando, caso exista, o powerUP escondido pelo mesmo),
além disto, nas posições onde se dá a explosão irá eliminar todos os jogadores que estejam nessa posição e, 
caso exista, irá colocar o tempo para explodir
da bomba dessa possição a 1(de modo a explodir no instante de tempo seguinte).
Para simular o efeito espiral de fechar o mapa criamos uma função que, 
recebendo o mapa e o tempo de jogo, coloca paredes de forma espiral
de modo a fechar o mapa e remove tudo que exista na posição que se deve fechar num dado instante de tempo.


\subsection{Tarefa5}

Para realizar esta tarefa utilizamos a função play da biblioteca Gloss e 
utilizamos o código realizado nas tarefas 1,2,4 e 6(com algumas alterações)
para implementar na função play.
Decidimos criar uma implementação simples do jogo, utilizando um mapa de dimensão 13, com 4 jogadores(1 humano, 3 bots) e tempo fixo igual a 4 minutos.

O trabalho realizado na tarefa1 foi utilizado para gerar o mapa inicial do jogo
O trabalho realizado na tarefa2 foi utilizado para simular o efeito de um comando do jogador,
que joga com o jogador '0' e utilizar as setas do teclado para os comandos 'D','U','L','R' e a tecla F1 para o comando B.
O trabalho realizado na tarefa4 é utilizado para simular o efeito do tempo.
O bot realizado na tarefa6 é utilizado para os 3 jogadores automáticos.


\subsection{Tarefa6}

Para a realização do nosso bot decidimos criar dois modos:

- Modo perigo: a posição onde o bot se encontra está em risco de ter uma explosão logo o bot apenas procura uma direcção para onde possa encontrar uma posição segura.

- Modo seguro: a posição onde o bot se encontra é segura(não está em risco de explosão) e então o bot irá verificar se existe algum proveito de plantar uma bomba, e caso exista e tenha um caminho seguro para se afastar planta a bomba, caso contrário apenas procura seguir o seu caminho para a powerUP visível mais proxima, ou caso não existam powerUPs visíveis segue o caminho que mais o aproxime do centro da grelha.


\pagebreak
\section{Conclusões}
\label{sec:conclusao}

No ínicio deste projecto foi nos proposto a implementação do clássico jogo Bomberman em battle mode, penso que no final deste projecto conseguimos
 implementar o jogo de uma forma, embora simples,prática, agradável e fácil de jogar.

Terminado este projecto penso que foram cumpridos os principais objectivos do mesmo que, mais do que implementar este jogo, eram,
introduzir-nos no mundo da programação, mais concretamente desenvolvimente de software e, familiarizar-nos com a linguagem haskell entre outras ferramentas uteis ao desenvolvimento de projectos.

\end{document}